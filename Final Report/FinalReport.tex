\documentclass[11pt]{report}
\usepackage{geometry}        % See geometry.pdf to learn the layout options. There are lots.
\geometry{letterpaper}          % ... or a4paper or a5paper or ... 
%\geometry{landscape}        % Activate for for rotated page geometry
\usepackage[parfill]{parskip}  % Activate to begin paragraphs with an empty line rather than an indent
\usepackage{graphicx}
\usepackage{amssymb}
\usepackage{epstopdf}
\DeclareGraphicsRule{.tif}{png}{.png}{`convert #1 `dirname #1`/`basename #1 .tif`.png}
\usepackage{qtree}
\usepackage{multicol}
\usepackage{listings}
\usepackage{color}
\definecolor{mygreen}{rgb}{0,0.6,0}
\definecolor{mygray}{rgb}{0.5,0.5,0.5}
\definecolor{mymauve}{rgb}{0.58,0,0.22}
\lstset{
 basicstyle=\footnotesize,
 language={[Objective]Caml},
 breaklines=true,
 tabsize=2,
 frame=single,
 numbers=left,
 title=\lstname,
 commentstyle=\color{mygreen},
 numberstyle=\small\color{mygray},
 stringstyle=\color{mymauve},
 showstringspaces=false,
 rulecolor=\color{black}
}

\title{ChemLAB \\
Final Report \\
COMS W4115 - Programming Languages \& Translators \\
Professor Stephen Edwards}
\author{Alice Chang (avc2120) \and Gabriel Lu (ggl2110) \and Martin Ong (mo2454)}
\date{\today}             % Activate to display a given date or no date

\begin{document}
\maketitle
\tableofcontents

\chapter*{Introduction}
\addcontentsline{toc}{chapter}{Introduction}
ChemLab is a language that will allow users to conveniently manipulate chemical elements. It can be used to solve chemistry and organic chemistry problems including, but not limited to, stoichiometeric calculations, oxidation-reduction reactions, acid-base reactions, gas stoichiometry, chemical equilibrium, thermodynamics, stereochemistry, and electrochemistry. It may also be used for intensive study of a molecule's properties such as chirality or aromaticity. These questions are mostly procedural and there is a general approach to solving each specific type of problem. For example, to determine the molecular formula of a compound: 1) use the mass percents and molar mass to determine the mass of each element present in 1 mole of compound 2) determine the number of moles of each element present in 1 mole of compound. Albeit these problems can generally be distilled down to a series of plug-and-chug math calculations, these calculations can become extremely tedious to work out by hand as molecules and compounds become more complex (imagine having to balance a chemical equation with Botox: $C_{6760}H_{10447}N_{1743}O_{2010}S_{32}$). Our language can be used to easily create programs to solve such problems through the use of our specially designed data types and utilities.

\chapter{Language Tutorial}
\section{Program Execution}
\texttt{make} creates an executable \texttt{chemlab}
To compile and run a \texttt{.chem} program, simply run the executable \texttt{chemlab} with your \texttt{.chem} file as the only argument.
\texttt{./chemlab <program name>.chem}

It then compiles the ChemLab file into Java bytecode, which is then executed on a Java virtual machine.

\section{Variables}
Variables in ChemLAB must be declared as a specific type. To use a variable, declare the type of the variable, and assign it to the value that you want like this:
 \begin{verbatim}
 int myNum = 5;
 String hello = "World"; 
 \end{verbatim}
\section{Control Flow}
ChemLAB supports \emph{if/else} statements: 
 \begin{verbatim}
 if(10>6){
 print("inside the if");
 else{
 print("inside the else");
 }
 \end{verbatim}
 
ChemLAB supports \emph{while loops}:
 \begin{verbatim}
 while(i > 0){
 print(i);
 i = i-1;
 }
 \end{verbatim}
 
\section{Functions}
Functions are the basis of ChemLAB. All programs in ChemLAB must contain one ``main'' function which is the starting point for the program. Functions can be passed any amount of parameters and are declared using the function keyword. The parameters within a function declaration must have type specifications. 

This is a function that takes in two parameters:
 \begin{verbatim}
 function main(int A, int B){
 	print A;
 }
 \end{verbatim}
 
This is a function that takes in no parameters:
 \begin{verbatim}
 function main(){
 print "Hello World";
 }
 \end{verbatim}

\section{Printing to stdout}

To print to stdout, simply use the built-in function \emph{print}
\begin{verbatim}
 print(6);
 print("Hello World");
\end{verbatim}

\chapter{Language Reference Manual}
\section{Types}
\subsection{Primitive Types}
There are four primitive types in ChemLab: boolean, int, double, and string.

\begin{description}
\item[Boolean] \hfill \\
The boolean data type has only two possible values: true and false. The boolean data type can be manipulated in boolean expressions involving the AND, OR, and NOT operators.

\item[Integers] \hfill \\
Much like in the Java programming language, the int data type is represented with 32-bits and in signed two's complement form. It has a minimum value of $-2^{31}$ and maximum value of $2^{31}$. There is no automatic type conversion between a variable of type int and of type double. In fact, an error will occur when the two primitive types are intermixed.

\item[Double] \hfill \\
Much like in the Java programming language, a double is a double-precision 64-bit IEEE 754 floating point with values ranging from $4.94065645841246544e-324d$ to $1.79769313486231570e+308d$ (positive or negative). Double should be used under any circumstance when there are decimal values.

\item[String] \hfill \\
Unlike in the C programming language, a string is a primitive type rather than a collection of characters. A string is a sequence of characters surrounded by double quotes ``"’’. Our language supports string concatenation. In the context of strings, the ``+'' operator concatenates two strings together to form a new string. 
\end{description}

\subsection{Non-Primitive Types}
The language comes built-in with lists, elements, molecules, equation.

\begin{description}
\item[Lists] \hfill \\
A list is a collection of items that maintains the order in which the items were added much like an ArrayList in Java. The type of items in a list must be declared and the type must remain consistent throughout the lifetime of the program. A list is declared in a syntax very similar to declaration in Java:
\begin{verbatim}<type> <identifier>[] = [ element_1, element_2, ......., element_n]\end{verbatim}

\item[Element] \hfill \\
Since there are only 118 elements, it could have been possible to hard code each element into the language. However, we chose not to do this to give the user a greater degree of flexibility in terms of declaring the properties of the element they want to consider because isotopes of elements have different amounts of neutrons and some elements can exist in more than one state. Element is declared with (atomic number, mass number, charge). The element type is the basic building block provided by the program that can be used to create molecules, compounds, etc. Elements are immutable.

$_6^{12}C$ is represented as: \texttt{element C(6, 12, 0);}

$_6^{14}C$ is represented as: \texttt{element C(6, 14, 0);}

\item[Molecule] \hfill \\ For the purpose of the language, there is no distinction between molecule or compound and both are declared the same way. A molecule is declared as a list of elements surrounded by braces.

$NaCl$ is represented as: \texttt{molecule NaCl \{[Na, Cl]\}}

\item[Equation] \hfill \\
Equation is declared in the following way: (list of elements/molecules on left side of reaction, list of elements/molecules on right side of reaction). Underneath, it is essentially, two lists that keep track of the two sides of the equation.

\texttt{<equationName>.right} or \texttt{<equationName>.left} allows easy access to one side of the equation. Once declared, an equation is immutable.

$NaOH + HCl \rightarrow NaCl +H_20$ is represented as:

\texttt{equation NaClReaction = \{[NaOH, HCl], [NaCl, H2O]\};}

\end{description}

\subsection{Type Inference}
The language is not type-inferred, making it necessary to explicitly declare types.

\section{Lexical Conventions}
\subsection{Identifiers}
An identifier is a sequence of letters or digits in which the first character must be a lowercase letter. Our language is case sensitive, so upper and lower case letters are considered different.

\subsection{Keywords}
The following identifiers start with a lowercase letter and are reserved for use as keywords, and may not be used otherwise:
\begin{multicols}{3}
 \begin{itemize}
 \item int
 \item double
 \item string
 \item boolean
 \item element
 \item molecule
 \item equation
 \item if
 \item else
 \item while
 \item function
 \item return
 \item true
 \item false
 \item print
 \item call
 \end{itemize}
\end{multicols}

\subsection{Literals}
Literals are values written in conventional form whose value is obvious. Unlike variables, literals do not change in value. An integer or double literal is a sequence of digits. A boolean literal has two possible values: true or false.

\subsection{Punctuation}
These following characters have their own syntactic and semantic significance and are not considered operators or identifiers.
\begin{center}
\begin{tabular}{ c|p{0.4\textwidth}|l }
 Punctuator & Use & Example \\
 \hline
 \texttt{,} & List separator, function parameters & \texttt{function int sum(int a, int b);} \\
 \texttt{;} & Statement end & \texttt{int x = 3;} \\
 \texttt{"} & String declaration & \texttt{string x = "hello";} \\
 \texttt{[]} & List delimiter & \texttt{int x[] = [1, 2, 3];} \\
 \texttt{\{\}} & Statement list deliminiter, and element/molecule/equation declaration & \texttt{if(expr) \{ statements \}} \\
 \texttt{()} & Conditional parameter delimiter, expression precedence & \texttt{while(i > 2)} \\
\end{tabular}
\end{center}

\subsection{Comments}
Much like in the C programming language, the characters \texttt{/*} introduce a comment, which terminates with the characters \texttt{*/}. Single line comments start with \texttt{//} and end at the new line character \texttt{$\backslash$n}.

\subsection{Operators}
\begin{center}
\begin{tabular}{ c|l|l }
Operator & Use & Associativity \\
\hline
\texttt{=} & Assignment    & Right \\
\texttt{==} & Test equivalence  & Left \\
\texttt{!=} & Test inequality   & Left \\
\texttt{>} & Greater than   & Left \\
\texttt{<} & Less than   & Left \\
\texttt{>=} & Greater than or equal to & Left \\
\texttt{<=} & Less than or equal to  & Left \\
\texttt{\&\&} & AND    & Left \\
\texttt{||} & OR    & Left \\
\texttt{.} & Access    & Left \\
\texttt{*} & Multiplication   & Left \\
\texttt{/} & Division    & Left \\
\texttt{+} & Addition    & Left \\
\texttt{-} & Subtraction   & Left \\
\texttt{\^} & Concatenate   & Left \\
\texttt{\%} & Modulo    & Left \\
\end{tabular}
\end{center}

The precedence of operators is as follows (from highest to lowest):
% \begin{multicols}{2}
\begin{enumerate}
 \item \texttt{* / \%}
 \item \texttt{+ -}
 \item \texttt{< > <= >=}
 \item \texttt{== !=}
 \item \texttt{\&\&}
 \item \texttt{||}
 \item \texttt{.}
 \item \texttt{\^}
 \item \texttt{=}
\end{enumerate}
% \end{multicols}

\section{Syntax}
A program in ChemLab consists of at least one function, where one of them is named ``main''. Within each function there is a sequence of zero or more valid ChemLab statements.

\subsection{Expressions}

An expression is a sequence of operators and operands that produce a value. Expressions have a type and a value and the operands of expressions must have compatible types. The order of evaluation of subexpressions depends on the precedence of the operators but, the subexpressions themselves are evaluated from left to right. 

\subsubsection{Constants}

Constants can either be of type boolean, string, int, or double. 

\subsubsection{Identifiers}

An identifier can identify a primitive type, non-primitive type, or a function. The type and value of the identifier is determined by its designation. The value of the identifier can change throughout the program, but the value that it can take on is restricted by the type of the identifier. Furthermore, after an identifier is declared, there can be no other identifiers of the same name declared within the scope of the whole program. 
\begin{verbatim}
int x = 3;
x = true; //syntax error
boolean x = 5; //error, x has already been declared 
\end{verbatim}

\subsubsection{Binary Operators}

Binary operators can be used in combination with variables and constants in order to create complex expressions. A binary operator is of the form : 
\texttt{<expression> <binary-operator> <expression>}

\begin{description}
\item[Arithmetic operators]
Arithmetic operators include \texttt{*}, \texttt{/}, \texttt{\%}, \texttt{+}, and \texttt{-}. The operands to an arithmetic operator must be numbers. the type of an arithmetic operator expression is either an int or a double and the value is the result of calculating the expression. Note, can not do arithmetic operations when the values involved are a mix of int and double.
	\begin{description}
		\item[\texttt{expression * expression}] \hfill \\ The binary operator \texttt{*} indicates multiplication. It must be performed between two int types or two double types. No other combinations are allowed. 
		\item[\texttt{expression / expression}] \hfill \\ The binary operator \texttt{/} indicates division. The same type considerations as for multiplication apply. 
		\item[\texttt{expression \% expression}] \hfill \\ The binary operator \texttt{\%} returns the remainder when the first expression is divided by the second expression. Modulo is only defined for int values that have a positive value. 
		\item[\texttt{expression + expression}] \hfill \\ The binary operator \texttt{+} indicates addition and returns the sum of the two expressions. The same type considerations as for multiplication apply. 
		\item[\texttt{expression - expression}] \hfill \\ The binary operator \texttt{-} indicates subtraction and returns the difference of the two expressions. The same type considerations as for multiplication apply.
	\end{description}

\item[Relational operators]
Relational operators include \texttt{<}, \texttt{>}, \texttt{<=}, \texttt{>=}, \texttt{==}, and \texttt{!=}. The type of a relational operator expression is a boolean and the value is true if the relation is true while it is false if the relation is false.
	\begin{description}
		\item[\texttt{expression1 > expression2}] \hfill \\ The overall expression returns true if expression1 is greater than expression 2
		\item[\texttt{expression1 < expression2}] \hfill \\ The overall expression returns true if expression1 is less than expression 2
		\item[\texttt{expression1 >= expression2}] \hfill \\ The overall expression returns true if expression1 is greater than or equal to expression 2
		\item[\texttt{expression1 <= expression2}] \hfill \\ The overall expression returns true if expression1 is less than or equal to expression 2
		\item[\texttt{expression1 == expression2}] \hfill \\ The overall expression returns true if expression1 is equal to expression 2. 
		\item[\texttt{expression1 != expression2}] \hfill \\ The overall expression returns true if expression1 is not equal to expression 2
	\end{description}

\item[Assignment operator]
The assignment operator (\texttt{=}) assigns whatever is on the right side of the operator to whatever is on the left side of the operator
	\begin{description}
		\item[\texttt{expression1 = expression2}] \hfill \\ expression1 now contains the value of expression2
	\end{description}

\item[Access operator]
 The access operator is of the form expression.value. The expression returns the value associated with the particular parameter. The expression must be of a non-primitive type. 

\item[Logical operators]
Logical operators include AND (\texttt{\&\&}) and OR (\texttt{||}). The operands to a logical operator must both be booleans and the result of the expression is also a boolean.
	\begin{description}
		\item[\texttt{expression1 \&\& expression2}] \hfill \\ The overall expression returns true if and only if expression1 evaluates to true and expression2 also evaluates to true. 
		\item[\texttt{expression1 || expression2}] \hfill \\ The overall expression returns true as long as expression1 and expression2 both do not evaluate to false. 
	\end{description}
\end{description}

\subsubsection{Parenthesized Expression}
Any expression surrounded by parentheses has the same type and value as it would without parentheses. The parentheses merely change the precedence in which operators are performed in the expression. 

\subsubsection{Function Creation}
The syntax for declaration of a function is as follows

\begin{verbatim}
function functionName (type parameter1, type parameter 2, ...) {
 statements
}
\end{verbatim}

The function keyword signifies that the expression is a function. Parameter declaration is surrounded by parentheses where the individual parameters are separated by commas. All statements in the function must be contained within the curly braces. A good programming practice in ChemLab is to declare all the functions at the beginning of the program so that the functions will definitely be recognized within the main of the program.

\subsubsection{Function Call}

Calling a function executes the function and blocks program execution until the function is completed. When a function is called, the types of the parameter passed into the function must be the same as those in the function declaration. The way to call a function is as follows using the Call keyword: 
call functionName(param1, param2, etc...) 
When a function with parameters is called, the parameters passed into the function are evaluated from left to right and copied by value into the function's scope.
functionName( ) if there are no parameters for the function

\subsection{Statements}

A statement in ChemLab does not produce a value and it does not have a type. An expression is not a valid statement in ChemLab. 

\subsubsection{Selection Statements}

A selection statement executes a set of statements based on the value of a specific expression. In ChemLab, the main type of selection statement is the if-else statement. An if-else statement has the following syntax:
\begin{verbatim}
 if( expression){

 }else{

 }
\end{verbatim}
Expression must evaluate to a value of type boolean. If the expression evaluates to true, then the statements within the first set of curly brackets is evaluated. If the expression evaluates to false, then the statements in the curly brackets following else is evaluated. If-else statements can be embedded within each other. Much like in the C programming language, the dangling if-else problem is resolved by assigning the else to the most recent else-less if. Unlike in Java, an if must be followed by an else. A statement with only if is not syntactically correct. 
\begin{verbatim}
if ( ){
 if ( ){

 }else{
 
 }
}else{

}
\end{verbatim}
\subsubsection{Iteration Statements}

ChemLab does not have a for loop unlike most programming languages. The only iteration statement is the while loop. The while statements evaluates an expression before going into the body of the loop. The expression must be of type boolean and the while loop while continue executing so long as the expression evaluates to true. Once the expression evaluates to false, the while loop terminates. The while loop syntax is as follows:
\begin{verbatim}
while ( expression ) {
 statements 
}
\end{verbatim}
Note that if values in the expression being evaluated are not altered through each iteration of the loop, there is a risk of going into an infinite loop. 

\subsubsection{Return Statements}

A return statement is specified with the keyword return. In a function, the expression that is returned must be of the type that the function has declared. The syntax of a return statement is:
 return expression;

The return statement will terminate the function it is embedded in or will end the entire program if it is not contained within a function. 

\subsection{Scope}

A block is a set of statements that get enclosed by braces. An identifier appearing within a block is only visible within that block. However, if there are two nested blocks, an identifier is recognizable and can be edited within the nested block. 
\begin{verbatim}
function int notRealMethod(int x){
int y = 4;
while(x>5){
 while(z>2){
 y++;
 }
}
\end{verbatim}
In this case, y is recognizable within the second while loop and its value will be incremented. One must also note that, functions only have access to those identifiers that are either declared within their body or are passed in as parameters.

\section{Built-in Functions}
\begin{description}
	\item[Balance Equations] \hfill \\
	Given an unbalanced equation, this utility will be able to compute the correct coefficients that go in front of each molecule to make it balanced

	\item[Molar Mass Calculation] \hfill \\
	Given a molecule, this utility will be able to compute the total molar mass of the molecule

	\item[Naming of Molecules] \hfill \\
	Given a molecule, the utility will print out the name in correct scientific notation (ex. $H_20$ will be printed as Dihydrogen Monoxide) 

	\item[Printing of Equations] \hfill \\
	Given an equation, the utility will print out the equation in correct scientific notation 

	\item[Amount of Moles] \hfill \\
	Given the element and the amount of grams of the element, this utility will return the amount of moles of the element.
\end{description}

\chapter{Project Plan}
Like any project, careful planning and organization is paramount to the success of the project. More importantly however, is the methodical execution of the plan. Although we originally developed a roadmap for success as well as implemented a number of project management systems, we did not follow the plan as intended. This section outlines our proposed plans for making ChemLAB happen and the actual process that we went through. 
\section{Proposed Plan}
We had originally planned to use the waterfall model in our software development process in which we would first develop a design for our language, followed by implementation, and finally testing. The idea was for all team members to dedicate complete focus to each stage in the project. Especially since we only had three members on our team, our roles were not as distinct and everyone had the chance to work, at least in some capacity, in all the roles. We intended to meet consistently each week on for at least two hours. During our meetings, each member was suppose to give an update about what he or she had been working on the past week as well as plans for the upcoming week and any challenges he or she faced that required the attention of the rest of the group. To help facilitate communication and the planning of meetings, we used Doodle to vote on what times were best for meetings. Also, in order to improve team dynamics, we planned to meet at least once every two weeks outside the context of school in order to hang out and have fun. Development would occur mostly on Mac OS and Windows 7, using the latest versions of OCaml, Ocamllex, and OCamlyacc for the compiler. We used Github for version control and makefiles to ease the work of compiling and testing code. The project timeline that we had laid out at the beginning was as follows:
\begin{itemize}
\item Sept 24th: Proposal Due Date
\item Oct 2nd: ChemLAB syntax roughly decided upon
\item Oct 23th: Scanner/Parser/AST unambiguous and working
\item Oct 27th: LRM Due Date
\item Nov 9th: Architectural design finalized
\item Dec 5th: Compile works, all tests passed
\item Dec 12th: Project report and slides completed 
\item Dec 17th: Final Project Due Date 
\end{itemize}

\section{What Actually Happened}
\begin{figure}[h]
 \centering
 \includegraphics[width=0.8\textwidth]{Github_Graph}
\end{figure}
This graph was pulled from Github reflecting the number of commits being made over the span of this semester. Due to schedule conflicts and a false sense of security, we did not start intensely working on the project until after Thanksgiving break. Since we did not coordinate the development of the Scanner, AST, and parser with the writing of the LRM, our language did not have as concrete a structure as we had hoped. Furthermore, we did not have enough time to implement some of the features in our language such as object-orientation or more built-in functions. As we were developing the software, we did make sure to allow testing at all steps in the design process. In the test script, we had identifiers for how far in the compilation process we wanted the program to run. Thus, we were able to maintain testing capabilities even before all of our code was ready. We discuss the testing procedure in more detail in a subsequent section. 
\section{Team Responsibilities}
This subsection describes the contributions made by each team member:
\begin{description}
\item[Project Proposal] Gabriel L/Alice C/Martin O
\item[Scanner] Gabriel L
\item[AST] Alice C/Gabriel L/Martin O
\item[Parser] Alice C/Martin O/Gabriel L
\item[LRM] Gabriel L
\item[Code Generation] Alice C
\item[Semantic Analyzer] Gabriel L/Martin O
\item[Testing] Martin O 
\item[Final Report] Gabriel L/Martin O
\end{description}

\section{Project Log}
See Appendix B.

\chapter{Architectural Design}
The architectural design of ChemLAB can be divided into the following steps
\begin{enumerate}
	\item Scanning
	\item Parsing
	\item Semantic Analysis
	\item Java code generation
	\item Running the Java code
\end{enumerate}

\section{Scanning}
The ChemLAB scanner tokenizes the input into ChemLAB readable units. This process involves discarding whitespaces and comments. At this stage, illegal character combinations are caught. The scanner was written with ocamllex. 

\section{Parsing and Abstract Syntax Tree}
The parser generates an abstract syntax tree based on the tokens that were provided by the scanner. Any syntax errors are caught here. The parser was written with ocamlyacc. 

\section{Semantic Analysis}
The semantic analyzer takes in the AST that was generated by the parser and checks the AST for type errors as well as to make sure that statements and expressions are written in a way that corresponds to the syntax defined by the language. A semantically checked AST (SAST) is not generated. If no errors are thrown, then we can assume that it is safe to use the AST to generate Java code. 

\section{Java Generation}
The module walks the AST and generates Java code corresponding to the program. All of the code is put into two Java files. One contains graphics and one contains everything else related to the program. The Java code is generated but not compiled. This needs to be done by the ChemLAB script which will run the javac command. 

\chapter{Test Plan}
\section{Introduction}
To ensure that one person's change and updates would not affect the changes others made previously, an automated test was put in place to run through all the tests to make sure everything that worked before still continued to work. Testing was done using a bash shell script to automate the process. The shell script compiles and runs all the test files and compares them with the expected output. Test cases were written to test individual components of the language such as arithmetic, conditional loops, printing, etc.

\section{Sample Test Cases}
\lstinputlisting[language={},caption={Hello World test}]{../files_test/test1.chem}
\lstinputlisting[language={},caption={Int and String Variable Assignment}]{../files_test/test2.chem}
\lstinputlisting[language={},caption={Arithmetic test}]{../files_test/test3.chem}
\lstinputlisting[language={},caption={String Concatenation}]{../files_test/test4.chem}
\lstinputlisting[language={},caption={If Condition}]{../files_test/test5.chem}
\lstinputlisting[language={},caption={Nested If Condition}]{../files_test/test6.chem}
\lstinputlisting[language={},caption={While Loop}]{../files_test/test7.chem}
\lstinputlisting[language={},caption={Draw}]{../files_test/test8.chem}
\lstinputlisting[language={},caption={Balance}]{../files_test/test9.chem}

\chapter{Lessons Learned}
\section{Alice Chang}
``Never have I spent so much time on so little code that does so much'' adequately sums up my experience this semester in Edward’s Programming Languages and Translator’s class. Indeed, it was a perpetual struggle at first to get the hang of OCaml, which was like no other language I had tackled in the past. Yeah sure, it was essentially java, but upside down and insides out. Initially I entered this class with little knowledge of how a parser or compiler worked. Composing a project proposal knowing so little felt like a clumsy and fruitless attempt to fly when we barely knew how to walk. Yet throughout the course of the semester, I’ve gained much more than knowledge to build a compiler but also skills to work in a team and most importantly the ability to reassure the heart at times of desperation that everything was going to be all right despite the rapidly approaching deadline.

Our team was one man (or woman) short as we had three members. Despite of the slight disadvantage, we learned to view it as a mixed blessing as it was easy to find time to meet up. However as we soon learned, three heads was not always better than one, only when put together did we slowly start to compose our compiler. We experimented with multiple ways of programming: The Lonely All-Nighter in which we all stayed up coding separate codes that worked individually but would not compile as a whole and eventually The Cozy Campfire solution in which one person was primarily in charge of coding and two people gathered around providing feedback. Yet the Cozy Campfire also had its downfalls too – lots of ideas being expressed simultaneously and very little progress. Essentially it was like two overly opinionated backseat drivers bickering back and forth while the driver sat in baffled silence. We had so many ideas going on at once that it was often difficult for the programmer to follow so eventually we broke down our ideas into small milestones and accomplished them through a step-by-step procedure.

As we near our presentation date, we’ve gotten closer as a team and learned to manage our time well, communicate with teammates, and decipher cryptic existing code. Like soldiers in combat, our team suffered through endless out-of-bounds errors and bonded through several panic attacks when GitHub repeated crashed on us. Yet at the end of this class we’ll have earned our wings to soar through parsers, interpreters, and compilers and wear with us these experiences like a badge noble achievement—that is, at least until next semester when we take another class that will once again challenge our late-night coding abilities. Yet undoubtedly, the lessons learned through this semester will stay with us beyond this class.

\section{Martin Ong}
The most important lesson I learned from this project is that communication between members of the group is paramount and GitHub can be your friend. Often times, the most difficult problem we encountered was trying to understand the code other members have written and be able to incorporate their code in our own work. At times, the lack of communication led to clashes in our work where a person would change code back to what they thought was working, when in fact they were undoing the work of another person. This was also due to our unfamiliarity with GitHub. Before this class, most of us had only used GitHub for individual projects, so when conflicts came up and we had to merge them, often times the response was to freak out. Resolving these conflicts on GitHub were not easy as changes another person made looked like it didn't belong there to the person resolving the conflict. The mantra then was to ``just make it work'', so sometimes progress another person made was disappeared in this way.

If we could do this over again, I would definitely split up the project clearly into concrete slices for each member to take ownership on, such as having one person be in charge of one file. This way one person could keep track of everything that still needs to be done for a particular file. We worked in a non-hierarchical way where we would meetup and code together on the same computer. This led to a decrease in productivity, even though everyone could understand what the code did in the end. Having one less person in our group also put us at a disadvantage, because, even though it made it easier to schedule meetings with each other, each of us had to do much more.

I would also create small milestone deadlines to complete throughout the semester to be more efficient. Since this is probably one of the largest coding projects we have ever done, it did not hit us until Thanksgiving break that there were much more than we had anticipated. I believe that if we put more effort in the beginning to get a good foundation for the ast, scanner and parser, it would be easier working with the other components.

I must say, I have learned a lot about working as a team under a coding environment. We have definitely learned and changed a lot through this project, both in terms of OCaml and working as a team.

\appendix
\chapter{Code Listing}
\lstinputlisting[caption={Abstract Syntax Tree (\texttt{ast.ml})}]{../ast.ml}
\lstinputlisting[caption={Scanner (\texttt{scanner.mll})}]{../scanner.mll}
\lstinputlisting[caption={Parser (\texttt{parser.mly})}]{../parser.mly}
\lstinputlisting[caption={Semantic Checker (\texttt{semantic.ml})}]{../semantic.ml}
\lstinputlisting[caption={Compiler, Code Generation (\texttt{compile.ml})}]{../compile.ml}
\lstinputlisting[caption={Top-level Executable (\texttt{chemlab.ml})}]{../chemlab.ml}
\lstinputlisting[caption={Test Script (\texttt{test.sh})}][language=bash]{../test.sh}

\chapter{Project Log}
\lstinputlisting[language={}]{projectlog.txt}

\end{document} 
